
% Default to the notebook output style

    


% Inherit from the specified cell style.




    
\documentclass[11pt]{article}

    
    
    \usepackage[T1]{fontenc}
    % Nicer default font (+ math font) than Computer Modern for most use cases
    \usepackage{mathpazo}

    % Basic figure setup, for now with no caption control since it's done
    % automatically by Pandoc (which extracts ![](path) syntax from Markdown).
    \usepackage{graphicx}
    % We will generate all images so they have a width \maxwidth. This means
    % that they will get their normal width if they fit onto the page, but
    % are scaled down if they would overflow the margins.
    \makeatletter
    \def\maxwidth{\ifdim\Gin@nat@width>\linewidth\linewidth
    \else\Gin@nat@width\fi}
    \makeatother
    \let\Oldincludegraphics\includegraphics
    % Set max figure width to be 80% of text width, for now hardcoded.
    \renewcommand{\includegraphics}[1]{\Oldincludegraphics[width=.8\maxwidth]{#1}}
    % Ensure that by default, figures have no caption (until we provide a
    % proper Figure object with a Caption API and a way to capture that
    % in the conversion process - todo).
    \usepackage{caption}
    \DeclareCaptionLabelFormat{nolabel}{}
    \captionsetup{labelformat=nolabel}

    \usepackage{adjustbox} % Used to constrain images to a maximum size 
    \usepackage{xcolor} % Allow colors to be defined
    \usepackage{enumerate} % Needed for markdown enumerations to work
    \usepackage{geometry} % Used to adjust the document margins
    \usepackage{amsmath} % Equations
    \usepackage{amssymb} % Equations
    \usepackage{textcomp} % defines textquotesingle
    % Hack from http://tex.stackexchange.com/a/47451/13684:
    \AtBeginDocument{%
        \def\PYZsq{\textquotesingle}% Upright quotes in Pygmentized code
    }
    \usepackage{upquote} % Upright quotes for verbatim code
    \usepackage{eurosym} % defines \euro
    \usepackage[mathletters]{ucs} % Extended unicode (utf-8) support
    \usepackage[utf8x]{inputenc} % Allow utf-8 characters in the tex document
    \usepackage{fancyvrb} % verbatim replacement that allows latex
    \usepackage{grffile} % extends the file name processing of package graphics 
                         % to support a larger range 
    % The hyperref package gives us a pdf with properly built
    % internal navigation ('pdf bookmarks' for the table of contents,
    % internal cross-reference links, web links for URLs, etc.)
    \usepackage{hyperref}
    \usepackage{longtable} % longtable support required by pandoc >1.10
    \usepackage{booktabs}  % table support for pandoc > 1.12.2
    \usepackage[inline]{enumitem} % IRkernel/repr support (it uses the enumerate* environment)
    \usepackage[normalem]{ulem} % ulem is needed to support strikethroughs (\sout)
                                % normalem makes italics be italics, not underlines
    

    
    
    % Colors for the hyperref package
    \definecolor{urlcolor}{rgb}{0,.145,.698}
    \definecolor{linkcolor}{rgb}{.71,0.21,0.01}
    \definecolor{citecolor}{rgb}{.12,.54,.11}

    % ANSI colors
    \definecolor{ansi-black}{HTML}{3E424D}
    \definecolor{ansi-black-intense}{HTML}{282C36}
    \definecolor{ansi-red}{HTML}{E75C58}
    \definecolor{ansi-red-intense}{HTML}{B22B31}
    \definecolor{ansi-green}{HTML}{00A250}
    \definecolor{ansi-green-intense}{HTML}{007427}
    \definecolor{ansi-yellow}{HTML}{DDB62B}
    \definecolor{ansi-yellow-intense}{HTML}{B27D12}
    \definecolor{ansi-blue}{HTML}{208FFB}
    \definecolor{ansi-blue-intense}{HTML}{0065CA}
    \definecolor{ansi-magenta}{HTML}{D160C4}
    \definecolor{ansi-magenta-intense}{HTML}{A03196}
    \definecolor{ansi-cyan}{HTML}{60C6C8}
    \definecolor{ansi-cyan-intense}{HTML}{258F8F}
    \definecolor{ansi-white}{HTML}{C5C1B4}
    \definecolor{ansi-white-intense}{HTML}{A1A6B2}

    % commands and environments needed by pandoc snippets
    % extracted from the output of `pandoc -s`
    \providecommand{\tightlist}{%
      \setlength{\itemsep}{0pt}\setlength{\parskip}{0pt}}
    \DefineVerbatimEnvironment{Highlighting}{Verbatim}{commandchars=\\\{\}}
    % Add ',fontsize=\small' for more characters per line
    \newenvironment{Shaded}{}{}
    \newcommand{\KeywordTok}[1]{\textcolor[rgb]{0.00,0.44,0.13}{\textbf{{#1}}}}
    \newcommand{\DataTypeTok}[1]{\textcolor[rgb]{0.56,0.13,0.00}{{#1}}}
    \newcommand{\DecValTok}[1]{\textcolor[rgb]{0.25,0.63,0.44}{{#1}}}
    \newcommand{\BaseNTok}[1]{\textcolor[rgb]{0.25,0.63,0.44}{{#1}}}
    \newcommand{\FloatTok}[1]{\textcolor[rgb]{0.25,0.63,0.44}{{#1}}}
    \newcommand{\CharTok}[1]{\textcolor[rgb]{0.25,0.44,0.63}{{#1}}}
    \newcommand{\StringTok}[1]{\textcolor[rgb]{0.25,0.44,0.63}{{#1}}}
    \newcommand{\CommentTok}[1]{\textcolor[rgb]{0.38,0.63,0.69}{\textit{{#1}}}}
    \newcommand{\OtherTok}[1]{\textcolor[rgb]{0.00,0.44,0.13}{{#1}}}
    \newcommand{\AlertTok}[1]{\textcolor[rgb]{1.00,0.00,0.00}{\textbf{{#1}}}}
    \newcommand{\FunctionTok}[1]{\textcolor[rgb]{0.02,0.16,0.49}{{#1}}}
    \newcommand{\RegionMarkerTok}[1]{{#1}}
    \newcommand{\ErrorTok}[1]{\textcolor[rgb]{1.00,0.00,0.00}{\textbf{{#1}}}}
    \newcommand{\NormalTok}[1]{{#1}}
    
    % Additional commands for more recent versions of Pandoc
    \newcommand{\ConstantTok}[1]{\textcolor[rgb]{0.53,0.00,0.00}{{#1}}}
    \newcommand{\SpecialCharTok}[1]{\textcolor[rgb]{0.25,0.44,0.63}{{#1}}}
    \newcommand{\VerbatimStringTok}[1]{\textcolor[rgb]{0.25,0.44,0.63}{{#1}}}
    \newcommand{\SpecialStringTok}[1]{\textcolor[rgb]{0.73,0.40,0.53}{{#1}}}
    \newcommand{\ImportTok}[1]{{#1}}
    \newcommand{\DocumentationTok}[1]{\textcolor[rgb]{0.73,0.13,0.13}{\textit{{#1}}}}
    \newcommand{\AnnotationTok}[1]{\textcolor[rgb]{0.38,0.63,0.69}{\textbf{\textit{{#1}}}}}
    \newcommand{\CommentVarTok}[1]{\textcolor[rgb]{0.38,0.63,0.69}{\textbf{\textit{{#1}}}}}
    \newcommand{\VariableTok}[1]{\textcolor[rgb]{0.10,0.09,0.49}{{#1}}}
    \newcommand{\ControlFlowTok}[1]{\textcolor[rgb]{0.00,0.44,0.13}{\textbf{{#1}}}}
    \newcommand{\OperatorTok}[1]{\textcolor[rgb]{0.40,0.40,0.40}{{#1}}}
    \newcommand{\BuiltInTok}[1]{{#1}}
    \newcommand{\ExtensionTok}[1]{{#1}}
    \newcommand{\PreprocessorTok}[1]{\textcolor[rgb]{0.74,0.48,0.00}{{#1}}}
    \newcommand{\AttributeTok}[1]{\textcolor[rgb]{0.49,0.56,0.16}{{#1}}}
    \newcommand{\InformationTok}[1]{\textcolor[rgb]{0.38,0.63,0.69}{\textbf{\textit{{#1}}}}}
    \newcommand{\WarningTok}[1]{\textcolor[rgb]{0.38,0.63,0.69}{\textbf{\textit{{#1}}}}}
    
    
    % Define a nice break command that doesn't care if a line doesn't already
    % exist.
    \def\br{\hspace*{\fill} \\* }
    % Math Jax compatability definitions
    \def\gt{>}
    \def\lt{<}
    % Document parameters
    \title{python-chess\_evaluation}
    
    
    

    % Pygments definitions
    
\makeatletter
\def\PY@reset{\let\PY@it=\relax \let\PY@bf=\relax%
    \let\PY@ul=\relax \let\PY@tc=\relax%
    \let\PY@bc=\relax \let\PY@ff=\relax}
\def\PY@tok#1{\csname PY@tok@#1\endcsname}
\def\PY@toks#1+{\ifx\relax#1\empty\else%
    \PY@tok{#1}\expandafter\PY@toks\fi}
\def\PY@do#1{\PY@bc{\PY@tc{\PY@ul{%
    \PY@it{\PY@bf{\PY@ff{#1}}}}}}}
\def\PY#1#2{\PY@reset\PY@toks#1+\relax+\PY@do{#2}}

\expandafter\def\csname PY@tok@w\endcsname{\def\PY@tc##1{\textcolor[rgb]{0.73,0.73,0.73}{##1}}}
\expandafter\def\csname PY@tok@c\endcsname{\let\PY@it=\textit\def\PY@tc##1{\textcolor[rgb]{0.25,0.50,0.50}{##1}}}
\expandafter\def\csname PY@tok@cp\endcsname{\def\PY@tc##1{\textcolor[rgb]{0.74,0.48,0.00}{##1}}}
\expandafter\def\csname PY@tok@k\endcsname{\let\PY@bf=\textbf\def\PY@tc##1{\textcolor[rgb]{0.00,0.50,0.00}{##1}}}
\expandafter\def\csname PY@tok@kp\endcsname{\def\PY@tc##1{\textcolor[rgb]{0.00,0.50,0.00}{##1}}}
\expandafter\def\csname PY@tok@kt\endcsname{\def\PY@tc##1{\textcolor[rgb]{0.69,0.00,0.25}{##1}}}
\expandafter\def\csname PY@tok@o\endcsname{\def\PY@tc##1{\textcolor[rgb]{0.40,0.40,0.40}{##1}}}
\expandafter\def\csname PY@tok@ow\endcsname{\let\PY@bf=\textbf\def\PY@tc##1{\textcolor[rgb]{0.67,0.13,1.00}{##1}}}
\expandafter\def\csname PY@tok@nb\endcsname{\def\PY@tc##1{\textcolor[rgb]{0.00,0.50,0.00}{##1}}}
\expandafter\def\csname PY@tok@nf\endcsname{\def\PY@tc##1{\textcolor[rgb]{0.00,0.00,1.00}{##1}}}
\expandafter\def\csname PY@tok@nc\endcsname{\let\PY@bf=\textbf\def\PY@tc##1{\textcolor[rgb]{0.00,0.00,1.00}{##1}}}
\expandafter\def\csname PY@tok@nn\endcsname{\let\PY@bf=\textbf\def\PY@tc##1{\textcolor[rgb]{0.00,0.00,1.00}{##1}}}
\expandafter\def\csname PY@tok@ne\endcsname{\let\PY@bf=\textbf\def\PY@tc##1{\textcolor[rgb]{0.82,0.25,0.23}{##1}}}
\expandafter\def\csname PY@tok@nv\endcsname{\def\PY@tc##1{\textcolor[rgb]{0.10,0.09,0.49}{##1}}}
\expandafter\def\csname PY@tok@no\endcsname{\def\PY@tc##1{\textcolor[rgb]{0.53,0.00,0.00}{##1}}}
\expandafter\def\csname PY@tok@nl\endcsname{\def\PY@tc##1{\textcolor[rgb]{0.63,0.63,0.00}{##1}}}
\expandafter\def\csname PY@tok@ni\endcsname{\let\PY@bf=\textbf\def\PY@tc##1{\textcolor[rgb]{0.60,0.60,0.60}{##1}}}
\expandafter\def\csname PY@tok@na\endcsname{\def\PY@tc##1{\textcolor[rgb]{0.49,0.56,0.16}{##1}}}
\expandafter\def\csname PY@tok@nt\endcsname{\let\PY@bf=\textbf\def\PY@tc##1{\textcolor[rgb]{0.00,0.50,0.00}{##1}}}
\expandafter\def\csname PY@tok@nd\endcsname{\def\PY@tc##1{\textcolor[rgb]{0.67,0.13,1.00}{##1}}}
\expandafter\def\csname PY@tok@s\endcsname{\def\PY@tc##1{\textcolor[rgb]{0.73,0.13,0.13}{##1}}}
\expandafter\def\csname PY@tok@sd\endcsname{\let\PY@it=\textit\def\PY@tc##1{\textcolor[rgb]{0.73,0.13,0.13}{##1}}}
\expandafter\def\csname PY@tok@si\endcsname{\let\PY@bf=\textbf\def\PY@tc##1{\textcolor[rgb]{0.73,0.40,0.53}{##1}}}
\expandafter\def\csname PY@tok@se\endcsname{\let\PY@bf=\textbf\def\PY@tc##1{\textcolor[rgb]{0.73,0.40,0.13}{##1}}}
\expandafter\def\csname PY@tok@sr\endcsname{\def\PY@tc##1{\textcolor[rgb]{0.73,0.40,0.53}{##1}}}
\expandafter\def\csname PY@tok@ss\endcsname{\def\PY@tc##1{\textcolor[rgb]{0.10,0.09,0.49}{##1}}}
\expandafter\def\csname PY@tok@sx\endcsname{\def\PY@tc##1{\textcolor[rgb]{0.00,0.50,0.00}{##1}}}
\expandafter\def\csname PY@tok@m\endcsname{\def\PY@tc##1{\textcolor[rgb]{0.40,0.40,0.40}{##1}}}
\expandafter\def\csname PY@tok@gh\endcsname{\let\PY@bf=\textbf\def\PY@tc##1{\textcolor[rgb]{0.00,0.00,0.50}{##1}}}
\expandafter\def\csname PY@tok@gu\endcsname{\let\PY@bf=\textbf\def\PY@tc##1{\textcolor[rgb]{0.50,0.00,0.50}{##1}}}
\expandafter\def\csname PY@tok@gd\endcsname{\def\PY@tc##1{\textcolor[rgb]{0.63,0.00,0.00}{##1}}}
\expandafter\def\csname PY@tok@gi\endcsname{\def\PY@tc##1{\textcolor[rgb]{0.00,0.63,0.00}{##1}}}
\expandafter\def\csname PY@tok@gr\endcsname{\def\PY@tc##1{\textcolor[rgb]{1.00,0.00,0.00}{##1}}}
\expandafter\def\csname PY@tok@ge\endcsname{\let\PY@it=\textit}
\expandafter\def\csname PY@tok@gs\endcsname{\let\PY@bf=\textbf}
\expandafter\def\csname PY@tok@gp\endcsname{\let\PY@bf=\textbf\def\PY@tc##1{\textcolor[rgb]{0.00,0.00,0.50}{##1}}}
\expandafter\def\csname PY@tok@go\endcsname{\def\PY@tc##1{\textcolor[rgb]{0.53,0.53,0.53}{##1}}}
\expandafter\def\csname PY@tok@gt\endcsname{\def\PY@tc##1{\textcolor[rgb]{0.00,0.27,0.87}{##1}}}
\expandafter\def\csname PY@tok@err\endcsname{\def\PY@bc##1{\setlength{\fboxsep}{0pt}\fcolorbox[rgb]{1.00,0.00,0.00}{1,1,1}{\strut ##1}}}
\expandafter\def\csname PY@tok@kc\endcsname{\let\PY@bf=\textbf\def\PY@tc##1{\textcolor[rgb]{0.00,0.50,0.00}{##1}}}
\expandafter\def\csname PY@tok@kd\endcsname{\let\PY@bf=\textbf\def\PY@tc##1{\textcolor[rgb]{0.00,0.50,0.00}{##1}}}
\expandafter\def\csname PY@tok@kn\endcsname{\let\PY@bf=\textbf\def\PY@tc##1{\textcolor[rgb]{0.00,0.50,0.00}{##1}}}
\expandafter\def\csname PY@tok@kr\endcsname{\let\PY@bf=\textbf\def\PY@tc##1{\textcolor[rgb]{0.00,0.50,0.00}{##1}}}
\expandafter\def\csname PY@tok@bp\endcsname{\def\PY@tc##1{\textcolor[rgb]{0.00,0.50,0.00}{##1}}}
\expandafter\def\csname PY@tok@fm\endcsname{\def\PY@tc##1{\textcolor[rgb]{0.00,0.00,1.00}{##1}}}
\expandafter\def\csname PY@tok@vc\endcsname{\def\PY@tc##1{\textcolor[rgb]{0.10,0.09,0.49}{##1}}}
\expandafter\def\csname PY@tok@vg\endcsname{\def\PY@tc##1{\textcolor[rgb]{0.10,0.09,0.49}{##1}}}
\expandafter\def\csname PY@tok@vi\endcsname{\def\PY@tc##1{\textcolor[rgb]{0.10,0.09,0.49}{##1}}}
\expandafter\def\csname PY@tok@vm\endcsname{\def\PY@tc##1{\textcolor[rgb]{0.10,0.09,0.49}{##1}}}
\expandafter\def\csname PY@tok@sa\endcsname{\def\PY@tc##1{\textcolor[rgb]{0.73,0.13,0.13}{##1}}}
\expandafter\def\csname PY@tok@sb\endcsname{\def\PY@tc##1{\textcolor[rgb]{0.73,0.13,0.13}{##1}}}
\expandafter\def\csname PY@tok@sc\endcsname{\def\PY@tc##1{\textcolor[rgb]{0.73,0.13,0.13}{##1}}}
\expandafter\def\csname PY@tok@dl\endcsname{\def\PY@tc##1{\textcolor[rgb]{0.73,0.13,0.13}{##1}}}
\expandafter\def\csname PY@tok@s2\endcsname{\def\PY@tc##1{\textcolor[rgb]{0.73,0.13,0.13}{##1}}}
\expandafter\def\csname PY@tok@sh\endcsname{\def\PY@tc##1{\textcolor[rgb]{0.73,0.13,0.13}{##1}}}
\expandafter\def\csname PY@tok@s1\endcsname{\def\PY@tc##1{\textcolor[rgb]{0.73,0.13,0.13}{##1}}}
\expandafter\def\csname PY@tok@mb\endcsname{\def\PY@tc##1{\textcolor[rgb]{0.40,0.40,0.40}{##1}}}
\expandafter\def\csname PY@tok@mf\endcsname{\def\PY@tc##1{\textcolor[rgb]{0.40,0.40,0.40}{##1}}}
\expandafter\def\csname PY@tok@mh\endcsname{\def\PY@tc##1{\textcolor[rgb]{0.40,0.40,0.40}{##1}}}
\expandafter\def\csname PY@tok@mi\endcsname{\def\PY@tc##1{\textcolor[rgb]{0.40,0.40,0.40}{##1}}}
\expandafter\def\csname PY@tok@il\endcsname{\def\PY@tc##1{\textcolor[rgb]{0.40,0.40,0.40}{##1}}}
\expandafter\def\csname PY@tok@mo\endcsname{\def\PY@tc##1{\textcolor[rgb]{0.40,0.40,0.40}{##1}}}
\expandafter\def\csname PY@tok@ch\endcsname{\let\PY@it=\textit\def\PY@tc##1{\textcolor[rgb]{0.25,0.50,0.50}{##1}}}
\expandafter\def\csname PY@tok@cm\endcsname{\let\PY@it=\textit\def\PY@tc##1{\textcolor[rgb]{0.25,0.50,0.50}{##1}}}
\expandafter\def\csname PY@tok@cpf\endcsname{\let\PY@it=\textit\def\PY@tc##1{\textcolor[rgb]{0.25,0.50,0.50}{##1}}}
\expandafter\def\csname PY@tok@c1\endcsname{\let\PY@it=\textit\def\PY@tc##1{\textcolor[rgb]{0.25,0.50,0.50}{##1}}}
\expandafter\def\csname PY@tok@cs\endcsname{\let\PY@it=\textit\def\PY@tc##1{\textcolor[rgb]{0.25,0.50,0.50}{##1}}}

\def\PYZbs{\char`\\}
\def\PYZus{\char`\_}
\def\PYZob{\char`\{}
\def\PYZcb{\char`\}}
\def\PYZca{\char`\^}
\def\PYZam{\char`\&}
\def\PYZlt{\char`\<}
\def\PYZgt{\char`\>}
\def\PYZsh{\char`\#}
\def\PYZpc{\char`\%}
\def\PYZdl{\char`\$}
\def\PYZhy{\char`\-}
\def\PYZsq{\char`\'}
\def\PYZdq{\char`\"}
\def\PYZti{\char`\~}
% for compatibility with earlier versions
\def\PYZat{@}
\def\PYZlb{[}
\def\PYZrb{]}
\makeatother


    % Exact colors from NB
    \definecolor{incolor}{rgb}{0.0, 0.0, 0.5}
    \definecolor{outcolor}{rgb}{0.545, 0.0, 0.0}



    
    % Prevent overflowing lines due to hard-to-break entities
    \sloppy 
    % Setup hyperref package
    \hypersetup{
      breaklinks=true,  % so long urls are correctly broken across lines
      colorlinks=true,
      urlcolor=urlcolor,
      linkcolor=linkcolor,
      citecolor=citecolor,
      }
    % Slightly bigger margins than the latex defaults
    
    \geometry{verbose,tmargin=1in,bmargin=1in,lmargin=1in,rmargin=1in}
    
    

    \begin{document}
    
    
    \maketitle
    
    

    
    \begin{Verbatim}[commandchars=\\\{\}]
{\color{incolor}In [{\color{incolor}1}]:} \PY{k+kn}{from} \PY{n+nn}{IPython}\PY{n+nn}{.}\PY{n+nn}{core}\PY{n+nn}{.}\PY{n+nn}{display} \PY{k}{import} \PY{n}{HTML}\PY{p}{,} \PY{n}{display}
        \PY{n}{HTML}\PY{p}{(}\PY{l+s+s2}{\PYZdq{}\PYZdq{}\PYZdq{}}
        \PY{l+s+s2}{\PYZlt{}style\PYZgt{}}
        \PY{l+s+s2}{svg }\PY{l+s+s2}{\PYZob{}}
        \PY{l+s+s2}{    width:40}\PY{l+s+s2}{\PYZpc{}}\PY{l+s+s2}{ !important;}
        \PY{l+s+s2}{    height:40}\PY{l+s+s2}{\PYZpc{}}\PY{l+s+s2}{ !important;}
        \PY{l+s+s2}{\PYZcb{}}
        
        \PY{l+s+s2}{.container }\PY{l+s+s2}{\PYZob{}}\PY{l+s+s2}{ }
        \PY{l+s+s2}{    width:100}\PY{l+s+s2}{\PYZpc{}}\PY{l+s+s2}{ !important;}
        \PY{l+s+s2}{\PYZcb{}}
        \PY{l+s+s2}{\PYZlt{}/style\PYZgt{}}
        \PY{l+s+s2}{\PYZdq{}\PYZdq{}\PYZdq{}}\PY{p}{)}
\end{Verbatim}


\begin{Verbatim}[commandchars=\\\{\}]
{\color{outcolor}Out[{\color{outcolor}1}]:} <IPython.core.display.HTML object>
\end{Verbatim}
            
    \begin{Verbatim}[commandchars=\\\{\}]
{\color{incolor}In [{\color{incolor}2}]:} \PY{c+c1}{\PYZsh{} \PYZpc{}autosave 0}
        \PY{o}{\PYZpc{}}\PY{k}{config} IPCompleter.greedy=True
\end{Verbatim}


    \section{Python-Chess Evaluierung}\label{python-chess-evaluierung}

In diesem Notebook wird die "python-chess-library", oder auch "chess
core" genannt, Bibliothek evaluiert. Der "chess core" soll Funktionen
zum Erstellen des Schachbretts, berechnen der erlaubten/möglichen Züge,
durchführen der Züge etc. bereitstellen.

Dabei wird jede notwendige Funktion getestet und dessen Realisierung
dokumentiert. Alle notwendigen Funktionen sind folgend gelistet.

\begin{itemize}
\tightlist
\item
  Schachbrett erstellen
\item
  Schachbrett als ASCII ausgeben
\item
  Züge auf dem Schachbrett ziehen
\item
  Erlaubte Züge berechnen
\item
  Auf Schach/Schachmatt prüfen
\item
  Testen, ob Rochade, En Passant als erlaubte Züge gelistet werden
\item
  \emph{(Optional)} Schachbrett als SVG in JupyterNotebook ausgeben
\item
  \emph{(Optional)} Möglichkeiten evaluieren, Schachbrett je nach
  Positionierungen eindeutig identifizierbar in CSV zu schreiben
\end{itemize}

Des Weiteren werden folgende Aktionen/Berechnungen auf deren
Umsetzbarkeit getestet. Dabei wird herausgefunden werden was nötig ist,
um die aufgelisteten Aktionen mit der Bibliothek durchzuführen.

\begin{itemize}
\tightlist
\item
  Berechnen \& Ausgeben erlaubter Züge für eine spezielle Figur
\item
  Abwechselnder User \& KI Input auf demselben Board
\item
  Berechnen des Wertes des Schachbretts und der attackierten Figuren
\item
  Speichern von Zügen/Entwicklung eines Schachbretts inklusive
  Sieg/Niederlagen als CSV-Datei
\end{itemize}

\subsection{Vorraussetzungen}\label{vorraussetzungen}

Folgend werden die allgemeinen Vorraussetzungen für die Verwendung und
Installation der "python-chess-library" erläutert.

\begin{itemize}
\tightlist
\item
  Python 3 Da die zu verwendende Bibliothek auf Python 3 basiert muss
  diese Version auf dem auszuführenden Computer vorhanden sein.

  \begin{itemize}
  \tightlist
  \item
    \href{https://wsvincent.com/install-python3-mac/}{macOS}
  \item
    \href{https://docs.python-guide.org/starting/install3/linux/}{Linux}
  \item
    \href{https://realpython.com/installing-python/}{Win}
  \end{itemize}
\item
  Jupyter Notebooks Damit die aufgeführten Scripts direkt im Browser
  ausgeführt werden können haben sich die Autoren dieser Arbeit darauf
  verständigt Jupyter Notebook zu nutzen. Dabei bietet diese Applikation
  den Vorteil, dass Code und Dokumente live geteilt werden können und
  der entwickelte Code sofort ausgeführt werden kann. Ebenfalls bietet
  Jupyter Notebook die Möglichkeit unter Anderem Daten zu visualisieren.

  \begin{itemize}
  \tightlist
  \item
    \href{https://jupyter.org/install}{Installationsanleitung}
  \end{itemize}
\item
  "python-chess-library"

  \begin{itemize}
  \tightlist
  \item
    \texttt{pip\ install\ python-chess} Im weiteren Verlauf dieses
    Notebooks werden zusätzliche Module, wie beispielsweise Pandas,
    genutzt, die ebenfalls auf dem auszuführenden Rechner installiert
    sein müssen.
  \end{itemize}
\end{itemize}

    \subsection{Erstellen eines Schachbretts und ausgeben als
ASCII}\label{erstellen-eines-schachbretts-und-ausgeben-als-ascii}

Zu Beginn muss die Python-Chess-Library eingebunden werden, die für den
weiteren Verlauf der Evaluierung und Implementierung benötigt wird.

    \begin{Verbatim}[commandchars=\\\{\}]
{\color{incolor}In [{\color{incolor}3}]:} \PY{k+kn}{import} \PY{n+nn}{chess}
\end{Verbatim}


    Das unten einzusehende Code Snippet zeigt, wie ein neues Schachbrett mit
standardmäßigen Positionierungen der Figuren erstellt werden kann. Dies
ist umzusetzen durch den Aufruf der \texttt{chess.Board()} Funktion und
das Speicherns der Rückgabe dieser Funktion in einer \texttt{board}
Variable. Dieses \texttt{board} beinhaltet die Positionierungen aller
Figuren und kann mittels der python-eigenen \texttt{print()} Funktion
als ASCII Code ausgegeben werden, wie unten einzusehen ist. Zuletzt wird
noch die Möglichkeit aufgezeigt, die auf einem spezifizierten Feld
befindliche Figur auszulesen. Dieses Feld wird dabei über das Datenfeld
\texttt{chess.B1} aufgerufen.

    \begin{Verbatim}[commandchars=\\\{\}]
{\color{incolor}In [{\color{incolor}4}]:} \PY{n}{board} \PY{o}{=} \PY{n}{chess}\PY{o}{.}\PY{n}{Board}\PY{p}{(}\PY{p}{)}
            
        \PY{n+nb}{print} \PY{p}{(}\PY{l+s+s2}{\PYZdq{}}\PY{l+s+se}{\PYZbs{}n}\PY{l+s+s2}{Board:}\PY{l+s+s2}{\PYZdq{}}\PY{p}{)}
        \PY{n+nb}{print} \PY{p}{(}\PY{n}{board}\PY{p}{)}
        
        \PY{n+nb}{print} \PY{p}{(}\PY{l+s+s2}{\PYZdq{}}\PY{l+s+se}{\PYZbs{}n}\PY{l+s+s2}{Piece at B1:}\PY{l+s+s2}{\PYZdq{}}\PY{p}{)}
        \PY{n+nb}{print} \PY{p}{(}\PY{n}{board}\PY{o}{.}\PY{n}{piece\PYZus{}at}\PY{p}{(}\PY{n}{chess}\PY{o}{.}\PY{n}{B1}\PY{p}{)}\PY{p}{)}
\end{Verbatim}


    \begin{Verbatim}[commandchars=\\\{\}]

Board:
r n b q k b n r
p p p p p p p p
. . . . . . . .
. . . . . . . .
. . . . . . . .
. . . . . . . .
P P P P P P P P
R N B Q K B N R

Piece at B1:
N

    \end{Verbatim}

    Dabei ist einzusehen, dass die Figuren durch einen Buchstaben abgekürzt
werden. Figuren des ersten Spielers werden als Kleinbuchstaben
dargestellt, Figuren des zweiten Spielers als äquivalente
Großbuchstaben. Leere Felder werden mit einem Punkt dargestellt. Aus
folgender Liste kann die Zurodnung der Buchstaben zu den Schachfiguren
entnommen werden:

\begin{itemize}
\tightlist
\item
  p/P: Pawn / Bauer
\item
  r/R: Rook / Turm
\item
  n/N: Knight / Springer
\item
  b/B: Bishop / Läufer
\item
  q/Q: Queen / Dame
\item
  k/K: King / König
\end{itemize}

    Die Zuordnung der Felder findet mittels einer Kombination aus einem
Buchstaben und einer Zahl statt. Die Buchstaben A-H geben dabei die
horizontale Reihe an, die Zahlen 1-8 die vertikalen Reihen. Jedem Feld
wird eine namensäquivalente Variable in der statischen Klasse "chess"
zugeordnet, wie beispielhaft in Zeile 16 im oberen Code Snippet erkannt
werden kann.

    \subsection{Alternierende Eingaben von Nutzer \&
KI}\label{alternierende-eingaben-von-nutzer-ki}

Damit ein Schachspiel zustande kommen kann, ist es zwingend
erforderlich, dass Nutzer und KI abwechselnd Züge auswählen und zum
Board "pushen" können. Dafür muss zunächst überprüft werden, welcher
Spieler an der Reihe ist und dieser muss einen Zug auswählen können, der
daraufhin von dem Schachbrett übernommen wird. Auf diesem Schachbrett
kann dann der nächste Spieler seinen Zug auswählen.

Bei der \emph{"python-chess"} Bibliothek kann der zu agierende Spieler
über die \texttt{board.turn} Variable ermittelt werden. Diese steht auf
\texttt{True} wenn Spieler 1 an der Reihe ist und auf \texttt{False},
wenn Spieler 2 den nächsten Zug machen muss. Beim Ausführen einer
\texttt{push} Operation auf dem \texttt{board}, wechselt die Variable
automatisch ihren Wert.

Damit das Spiel erkennt, wann dieses vorbei ist, bietet die Bibliothek
die \texttt{board.is\_game\_over()} Funktion. Diese gibt den Wert
\texttt{True} zurück, falls das Spiel auf Grund eines Schachmatts oder
anderer spielbeendender Umstände vorbei ist.

Um alle möglichen Züge ausgeben zu können, bietet die verwendete
Bibliothek die Funktion \texttt{board.legal\_moves}. Diese gibt alle
gültigen Züge nach \emph{chess960} Standard aus. Dabei werden nur die
Züge ausgegeben, die zum einen vom aktuellen Schachbrett aus
durchführbar sind und zum anderen nicht zu einer unmittelbaren
Niederlage führen. Das bedeutet, dass bei den legalen Zügen keine Züge
ausgegeben werden, bei der sich der Spieler beispielsweise selbst ins
"Schach" stellt oder ein durch den Gegner verursachtes "Schach"
ignoriert. Mittels der \texttt{board.uci()} Funktion können diese in
die, aus 4 Zeichen bestehende, leserlichere Form gebracht werden. Die
ersten beiden Zeichen stellen dabei das startende Feld dar, während die
zweiten zwei Zeichen das Feld darstellen, auf das sich die Figur vom
Startfeld bewegt.

Diese Funktion kann beim Zug des Nutzers verwendet werden, um diesem
eine einfachere Übersicht über seine Möglichkeiten zu geben und den Zug
in einer ihm verständlichen Form einzulesen.

Mittels der \texttt{list()"} Funktion können die legalen Züge zu einer
Liste zusammengefasst werden.

Im folgenden Beispiel sind alle notwendigen Schritte für ein Schachspiel
zwischen Nutzer und KI erkennbar. Die KI ermittelt dabei ihren Zug durch
eine zufällige Auswahl aus der Liste aller legalen Züge. Dabei wurden
die einzelnen Methoden, wie oben beschrieben, implementiert und genutzt.

Zusätzlich müssen einigen Funktionen der Python-Chess-Library importiert
werden, die für die Visualisierung eines Schachbretts als SVG benötigt
werden. Ebenfalls wird das Modul \texttt{random} eingebunden, da es für
die Berechnung zufälliger Züge benötigt wird.

    \begin{Verbatim}[commandchars=\\\{\}]
{\color{incolor}In [{\color{incolor}5}]:} \PY{k+kn}{import} \PY{n+nn}{chess}\PY{n+nn}{.}\PY{n+nn}{svg}
        \PY{k+kn}{import} \PY{n+nn}{random}
        \PY{n}{board} \PY{o}{=} \PY{n}{chess}\PY{o}{.}\PY{n}{Board}\PY{p}{(}\PY{p}{)}
\end{Verbatim}


    \texttt{get\_random\_move} wählt zufällig einen Zug aus den erlaubten
Zügen aus und gibt diese als \texttt{CHESS.Move} zurück. Hingegen wird
die Funktion \texttt{get\_legal\_moves\_uci} genutzt um die erlaubten
Züge berechnen zu lassen.

    \begin{Verbatim}[commandchars=\\\{\}]
{\color{incolor}In [{\color{incolor}6}]:} \PY{k}{def} \PY{n+nf}{get\PYZus{}random\PYZus{}move}\PY{p}{(}\PY{p}{)}\PY{p}{:}
            \PY{k}{return} \PY{n}{random}\PY{o}{.}\PY{n}{choice}\PY{p}{(}\PY{n+nb}{list}\PY{p}{(}\PY{n}{board}\PY{o}{.}\PY{n}{legal\PYZus{}moves}\PY{p}{)}\PY{p}{)}
        
        \PY{k}{def} \PY{n+nf}{get\PYZus{}legal\PYZus{}moves\PYZus{}uci}\PY{p}{(}\PY{p}{)}\PY{p}{:}
            \PY{k}{return} \PY{n+nb}{list}\PY{p}{(}\PY{n+nb}{map}\PY{p}{(}\PY{n}{board}\PY{o}{.}\PY{n}{uci}\PY{p}{,} \PY{n}{board}\PY{o}{.}\PY{n}{legal\PYZus{}moves}\PY{p}{)}\PY{p}{)}
\end{Verbatim}


    Die Funktion \texttt{get\_user\_move} wird verwendet um den Nutzer die
möglichen, bzw. erlaubten, Züge auszugeben. Dabei wird die zuvor
eingeführte Funktion zur Berechnung von legalen Zügen verwendet.

    \begin{Verbatim}[commandchars=\\\{\}]
{\color{incolor}In [{\color{incolor}7}]:} \PY{k}{def} \PY{n+nf}{get\PYZus{}user\PYZus{}move}\PY{p}{(}\PY{p}{)}\PY{p}{:}
            \PY{n+nb}{print}\PY{p}{(}\PY{l+s+s2}{\PYZdq{}}\PY{l+s+s2}{Possible Moves: }\PY{l+s+s2}{\PYZdq{}}\PY{p}{)}
            \PY{n+nb}{print}\PY{p}{(}\PY{n}{get\PYZus{}legal\PYZus{}moves\PYZus{}uci}\PY{p}{(}\PY{p}{)}\PY{p}{)}
            
            \PY{n+nb}{print}\PY{p}{(}\PY{l+s+s2}{\PYZdq{}}\PY{l+s+s2}{Enter your move:}\PY{l+s+s2}{\PYZdq{}}\PY{p}{)}
            \PY{n}{move} \PY{o}{=} \PY{n+nb}{input}\PY{p}{(}\PY{p}{)}
                
            \PY{k}{return} \PY{n}{chess}\PY{o}{.}\PY{n}{Move}\PY{o}{.}\PY{n}{from\PYZus{}uci}\PY{p}{(}\PY{n}{move}\PY{p}{)}
\end{Verbatim}


    Der folgende Codeausschnitt wird genutzt um einen Nutzer gegen den
Computer spielen zu lassen, der jeweils einen zufälligen Zug verwendet.
Hierbei wird für das Jupyter Notebook ein Counter eingeführt, damit das
Spiel bereits nach zwei Zügen pro Spieler beendet ist. Alternativ dazu
wird das Spiel beendet, sobald ein Sieger oder ein Patt feststeht.

Ebenfalls wird als Unterstützung für den Spieler pro Zug das Schachbrett
und die legalen Züge ausgegeben.

    \begin{Verbatim}[commandchars=\\\{\}]
{\color{incolor}In [{\color{incolor}8}]:} \PY{n}{counter} \PY{o}{=} \PY{l+m+mi}{0}
        \PY{k}{while} \PY{p}{(}\PY{o+ow}{not} \PY{n}{board}\PY{o}{.}\PY{n}{is\PYZus{}game\PYZus{}over}\PY{p}{(}\PY{p}{)} \PY{o+ow}{and} \PY{n}{counter} \PY{o}{\PYZlt{}} \PY{l+m+mi}{4}\PY{p}{)}\PY{p}{:}
            \PY{n+nb}{print}\PY{p}{(}\PY{l+s+s2}{\PYZdq{}}\PY{l+s+s2}{\PYZhy{}\PYZhy{}\PYZhy{}\PYZhy{}\PYZhy{}\PYZhy{}\PYZhy{}\PYZhy{}\PYZhy{}\PYZhy{}\PYZhy{}\PYZhy{}\PYZhy{}\PYZhy{}\PYZhy{}}\PY{l+s+s2}{\PYZdq{}}\PY{p}{)}
            \PY{n+nb}{print}\PY{p}{(}\PY{n}{board}\PY{p}{)}
            \PY{n+nb}{print}\PY{p}{(}\PY{l+s+s2}{\PYZdq{}}\PY{l+s+s2}{\PYZhy{}\PYZhy{}\PYZhy{}\PYZhy{}\PYZhy{}\PYZhy{}\PYZhy{}\PYZhy{}\PYZhy{}\PYZhy{}\PYZhy{}\PYZhy{}\PYZhy{}\PYZhy{}\PYZhy{}}\PY{l+s+s2}{\PYZdq{}}\PY{p}{)}
            \PY{n+nb}{print}\PY{p}{(}\PY{p}{)}
            
            \PY{k}{if} \PY{n}{board}\PY{o}{.}\PY{n}{turn}\PY{p}{:}
                \PY{n}{board}\PY{o}{.}\PY{n}{push}\PY{p}{(}\PY{n}{get\PYZus{}user\PYZus{}move}\PY{p}{(}\PY{p}{)}\PY{p}{)}
                \PY{n+nb}{print}\PY{p}{(}\PY{l+s+s2}{\PYZdq{}}\PY{l+s+s2}{Your Move: }\PY{l+s+s2}{\PYZdq{}}\PY{p}{)}
            \PY{k}{else}\PY{p}{:}
                \PY{n}{board}\PY{o}{.}\PY{n}{push}\PY{p}{(}\PY{n}{get\PYZus{}random\PYZus{}move}\PY{p}{(}\PY{p}{)}\PY{p}{)}
                \PY{n+nb}{print}\PY{p}{(}\PY{l+s+s2}{\PYZdq{}}\PY{l+s+s2}{AIs Move:}\PY{l+s+s2}{\PYZdq{}}\PY{p}{)}
                
            \PY{n}{counter}\PY{o}{+}\PY{o}{=}\PY{l+m+mi}{1}
            
        \PY{n+nb}{print}\PY{p}{(}\PY{n}{board}\PY{p}{)}
        \PY{n+nb}{print}\PY{p}{(}\PY{l+s+s2}{\PYZdq{}}\PY{l+s+s2}{[...]}\PY{l+s+s2}{\PYZdq{}}\PY{p}{)}
\end{Verbatim}


    \begin{Verbatim}[commandchars=\\\{\}]
---------------
r n b q k b n r
p p p p p p p p
. . . . . . . .
. . . . . . . .
. . . . . . . .
. . . . . . . .
P P P P P P P P
R N B Q K B N R
---------------

Possible Moves: 
['g1h3', 'g1f3', 'b1c3', 'b1a3', 'h2h3', 'g2g3', 'f2f3', 'e2e3', 'd2d3', 'c2c3', 'b2b3', 'a2a3', 'h2h4', 'g2g4', 'f2f4', 'e2e4', 'd2d4', 'c2c4', 'b2b4', 'a2a4']
Enter your move:
c2c3
Your Move: 
---------------
r n b q k b n r
p p p p p p p p
. . . . . . . .
. . . . . . . .
. . . . . . . .
. . P . . . . .
P P . P P P P P
R N B Q K B N R
---------------

AIs Move:
---------------
r . b q k b n r
p p p p p p p p
. . n . . . . .
. . . . . . . .
. . . . . . . .
. . P . . . . .
P P . P P P P P
R N B Q K B N R
---------------

Possible Moves: 
['g1h3', 'g1f3', 'd1a4', 'd1b3', 'd1c2', 'b1a3', 'c3c4', 'h2h3', 'g2g3', 'f2f3', 'e2e3', 'd2d3', 'b2b3', 'a2a3', 'h2h4', 'g2g4', 'f2f4', 'e2e4', 'd2d4', 'b2b4', 'a2a4']
Enter your move:
d1a4
Your Move: 
---------------
r . b q k b n r
p p p p p p p p
. . n . . . . .
. . . . . . . .
Q . . . . . . .
. . P . . . . .
P P . P P P P P
R N B . K B N R
---------------

AIs Move:
r . b q k b n r
. p p p p p p p
p . n . . . . .
. . . . . . . .
Q . . . . . . .
. . P . . . . .
P P . P P P P P
R N B . K B N R
[{\ldots}]

    \end{Verbatim}

    \subsection{Ausgeben des Boards als
SVG}\label{ausgeben-des-boards-als-svg}

Um eine bessere Darstellung und Erklärung der Implementation in der
theoretischen Ausarbeitung zu ermöglichen, wäre eine visuelle
Darstellung des Schachbretts wünschenswert. Die "python-chess-library"
ermöglicht dies durch ein Konvertieren des Boards zu einer SVG-Datei.
Diese kann dann mittels der python-eigenen \texttt{SVG()} Funktion im
Python-Notebook angezeigt werden.

Um dies zu ermöglichen, muss zunächst die SVG library aus
\emph{"IPython"} importiert werden. Anschließend kann das Board über die
\texttt{chess.svg.board()} Funktion konvertiert und anschließend als SVG
ausgegeben werden.

    \begin{Verbatim}[commandchars=\\\{\}]
{\color{incolor}In [{\color{incolor}9}]:} \PY{k+kn}{from} \PY{n+nn}{IPython}\PY{n+nn}{.}\PY{n+nn}{display} \PY{k}{import} \PY{n}{SVG}
        
        \PY{n}{board} \PY{o}{=} \PY{n}{chess}\PY{o}{.}\PY{n}{Board}\PY{p}{(}\PY{p}{)}
        \PY{n}{SVG}\PY{p}{(}\PY{n}{chess}\PY{o}{.}\PY{n}{svg}\PY{o}{.}\PY{n}{board}\PY{p}{(}\PY{n}{board}\PY{o}{=}\PY{n}{board}\PY{p}{)}\PY{p}{)} 
\end{Verbatim}

\texttt{\color{outcolor}Out[{\color{outcolor}9}]:}
    
    \begin{center}
    \adjustimage{max size={0.9\linewidth}{0.9\paperheight}}{output_18_0.pdf}
    \end{center}
    { \hspace*{\fill} \\}
    

    Die verwendete Bibliothek bringt den Vorteil mit sich, dass einige
wichtige Eigenschaften während eines Spiels visualisiert werden können.
Ein Beispiel für diese Visualisierung ist das Schachmatt, bei dem der
betroffende König rot hervorgehoben wird. Ein weiteres Beispiel zum
Thema Visualisierung wird im folgenden Kapitel gegeben.

Um diesen Fall simulieren zu können, wird in diesem Beispiel eine
bestimmte Situation während eines Schachspiels importiert, in der ein
Schachmatt vorliegt. Das Schachmatt auf der Position \texttt{E8} wird in
diesem Codeausschnitt manuell, durch den Übergabeparamter
\texttt{chess.E8}, hervorgehoben.

    \begin{Verbatim}[commandchars=\\\{\}]
{\color{incolor}In [{\color{incolor}10}]:} \PY{n}{board} \PY{o}{=} \PY{n}{chess}\PY{o}{.}\PY{n}{Board}\PY{p}{(}\PY{l+s+s2}{\PYZdq{}}\PY{l+s+s2}{1R2k3/R7/8/8/8/8/8/4K3 b KQkq \PYZhy{} 0 1}\PY{l+s+s2}{\PYZdq{}}\PY{p}{)}
         \PY{n}{SVG}\PY{p}{(}\PY{n}{chess}\PY{o}{.}\PY{n}{svg}\PY{o}{.}\PY{n}{board}\PY{p}{(}\PY{n}{board}\PY{p}{,} \PY{n}{check}\PY{o}{=}\PY{n}{chess}\PY{o}{.}\PY{n}{E8}\PY{p}{)}\PY{p}{)}
\end{Verbatim}

\texttt{\color{outcolor}Out[{\color{outcolor}10}]:}
    
    \begin{center}
    \adjustimage{max size={0.9\linewidth}{0.9\paperheight}}{output_20_0.pdf}
    \end{center}
    { \hspace*{\fill} \\}
    

    \subsection{Berechnen \& Ausgeben legaler Züge eines speziellen
Felds}\label{berechnen-ausgeben-legaler-zuxfcge-eines-speziellen-felds}

Für eine bessere, visuelle Darstellung in der theoretischen Ausarbeitung
wurde die Möglichkeit geprüft spezielle Felder auf dem ausgegebenen SVG
markieren zu können. Solche Felder können beispielsweise die errechneten
erlaubten Züge einer speziellen Figur / eines speziellen Felds
darstellen, um anzuzeigen, welche Möglichkeiten eine Figur in der
aktuellen Situation besitzt.

Dies kann realisiert werden, indem auf die legalen Züge zurückgegriffen
wird und diese gefiltert werden. Bei dem verwendeten Filter müssen die
Ausgangspositionen, die mit \texttt{move.from\_square} ausgelesen werden
können, mit dem eingegebenen Feld übereinstimmen.

    \begin{Verbatim}[commandchars=\\\{\}]
{\color{incolor}In [{\color{incolor}11}]:} \PY{n}{board} \PY{o}{=} \PY{n}{chess}\PY{o}{.}\PY{n}{Board}\PY{p}{(}\PY{p}{)}
         
         \PY{n+nb}{print}\PY{p}{(}\PY{l+s+s2}{\PYZdq{}}\PY{l+s+s2}{Please enter field:}\PY{l+s+s2}{\PYZdq{}}\PY{p}{)}
         \PY{n}{field} \PY{o}{=} \PY{n+nb}{input}\PY{p}{(}\PY{p}{)}
         
         \PY{n}{moves\PYZus{}from\PYZus{}spec\PYZus{}field} \PY{o}{=} \PY{n+nb}{list}\PY{p}{(}\PY{n+nb}{filter}\PY{p}{(}\PY{k}{lambda} \PY{n}{move}\PY{p}{:} \PY{n}{move}\PY{o}{.}\PY{n}{from\PYZus{}square} \PY{o+ow}{is} \PY{n}{chess}\PY{o}{.}\PY{n}{SQUARE\PYZus{}NAMES}\PY{o}{.}\PY{n}{index}\PY{p}{(}\PY{n}{field}\PY{p}{)}\PY{p}{,} \PY{n}{board}\PY{o}{.}\PY{n}{legal\PYZus{}moves}\PY{p}{)}\PY{p}{)}
\end{Verbatim}


    \begin{Verbatim}[commandchars=\\\{\}]
Please enter field:
a2

    \end{Verbatim}

    Diese herausgefilterten Züge werden dann zu deren Zielfelder zugeordnet.
Diese können mit der Funktion \texttt{move.to\_square} ausgelsen werden.

    \begin{Verbatim}[commandchars=\\\{\}]
{\color{incolor}In [{\color{incolor}12}]:} \PY{n}{square\PYZus{}nums} \PY{o}{=} \PY{n+nb}{list}\PY{p}{(}\PY{n+nb}{map}\PY{p}{(}\PY{k}{lambda} \PY{n}{move}\PY{p}{:} \PY{n}{move}\PY{o}{.}\PY{n}{to\PYZus{}square}\PY{p}{,} \PY{n}{moves\PYZus{}from\PYZus{}spec\PYZus{}field}\PY{p}{)}\PY{p}{)}
\end{Verbatim}


    Nun kann ein SquareSet erstellt und alle gefilterten Felder diesem
hinzugefügt werden. Das SquareSet kann dann beim Erstellen des Board
angegeben werden. Dies veranlasst das Zeichnen von Kreuzen auf den
berechneten Feldern, die von dem angegebenen Feld aus erreicht werden
können.

    \begin{Verbatim}[commandchars=\\\{\}]
{\color{incolor}In [{\color{incolor}13}]:} \PY{n}{squares} \PY{o}{=} \PY{n}{chess}\PY{o}{.}\PY{n}{SquareSet}\PY{p}{(}\PY{p}{)}
         \PY{k}{for} \PY{n}{square\PYZus{}num} \PY{o+ow}{in} \PY{n}{square\PYZus{}nums} \PY{p}{:} \PY{n}{squares}\PY{o}{.}\PY{n}{add}\PY{p}{(}\PY{n}{square\PYZus{}num}\PY{p}{)}
         
         \PY{n}{SVG}\PY{p}{(}\PY{n}{chess}\PY{o}{.}\PY{n}{svg}\PY{o}{.}\PY{n}{board}\PY{p}{(}\PY{n}{board}\PY{o}{=}\PY{n}{board}\PY{p}{,} \PY{n}{squares}\PY{o}{=}\PY{n}{squares}\PY{p}{)}\PY{p}{)}
\end{Verbatim}

\texttt{\color{outcolor}Out[{\color{outcolor}13}]:}
    
    \begin{center}
    \adjustimage{max size={0.9\linewidth}{0.9\paperheight}}{output_26_0.pdf}
    \end{center}
    { \hspace*{\fill} \\}
    

    \subsection{Speicherung der einzelnen Spielzüge in
history.csv}\label{speicherung-der-einzelnen-spielzuxfcge-in-history.csv}

Um die KI Entscheidungen um ihre Spielzüge auch von vorherigen Spielen
und desse Ausgängen abhängig zu machen, muss eine Historie angelegt
werden, die die Spielbretter und den Ausgang des Spiels im Nachgang
beinhaltet. Anhand dieser Historie kann die KI in zukünfigen Spielen
dann den Wert des Spielbrettes abschätzen und das Spielbrett evaluieren,
das die höchste Chance auf einen Sieg bietet.

Zur Erweiterung der Liste wird zuerst das Modul Pandas genutzt, das es
ermöglicht Informationen als Datenbank in einer Variable zu speichern.
Ebenso wird angegeben unter welchem Pfad sich die Zughistorie befinden
soll.

    \begin{Verbatim}[commandchars=\\\{\}]
{\color{incolor}In [{\color{incolor}14}]:} \PY{k+kn}{import} \PY{n+nn}{pandas} \PY{k}{as} \PY{n+nn}{pd}
         \PY{n}{HISTORY\PYZus{}FILE\PYZus{}LOC} \PY{o}{=} \PY{l+s+s2}{\PYZdq{}}\PY{l+s+s2}{res/history.csv}\PY{l+s+s2}{\PYZdq{}}
\end{Verbatim}


    Dazu kann testweise ein Spiel durchgespielt werden, wobei jeder Zug
zufällig aus der Liste der legalen Züge gewählt wird.

    \begin{Verbatim}[commandchars=\\\{\}]
{\color{incolor}In [{\color{incolor}15}]:} \PY{k}{def} \PY{n+nf}{get\PYZus{}random\PYZus{}move}\PY{p}{(}\PY{n}{board}\PY{p}{)}\PY{p}{:}
             \PY{k}{return} \PY{n}{random}\PY{o}{.}\PY{n}{choice}\PY{p}{(}\PY{n+nb}{list}\PY{p}{(}\PY{n}{board}\PY{o}{.}\PY{n}{legal\PYZus{}moves}\PY{p}{)}\PY{p}{)}
\end{Verbatim}


    Nach jedem Zug wird das Spielbrett in der "FEN" Darstellung einer Liste
hinzugefügt. Die "FEN" Darstellung ist eine gekürzte Form der
Darstellung des Spielbretts, wobei das Spielbrett dennoch eindeutig
identifizierbar bleibt. Dabei wird neben dem aktuellen Spielbrett auch
die Farbe der nächste zu spielenden Figur, die Anzahl der Züge beider
Seiten und weitere Informationen angegeben. Um diese aus der Darstellung
zu kürzen, wird die auf die "FEN" Darstellung des Boards die
\texttt{split} Funktion angewandt und nur das erste Element aus der
daraus entstehenden Liste gespeichert, da die einzelnen Merkmale in
dieser Darstellung per Leerzeichen getrennt werden.

    \begin{Verbatim}[commandchars=\\\{\}]
{\color{incolor}In [{\color{incolor}16}]:} \PY{k}{def} \PY{n+nf}{play\PYZus{}random\PYZus{}chess\PYZus{}game}\PY{p}{(}\PY{n}{board}\PY{p}{)}\PY{p}{:}
             \PY{n}{turn\PYZus{}list} \PY{o}{=} \PY{n+nb}{list}\PY{p}{(}\PY{p}{)}
             \PY{k}{while} \PY{o+ow}{not} \PY{n}{board}\PY{o}{.}\PY{n}{is\PYZus{}game\PYZus{}over}\PY{p}{(}\PY{p}{)}\PY{p}{:}
                 \PY{n}{turn\PYZus{}list}\PY{o}{.}\PY{n}{append}\PY{p}{(}\PY{n}{board}\PY{o}{.}\PY{n}{fen}\PY{p}{(}\PY{p}{)}\PY{o}{.}\PY{n}{split}\PY{p}{(}\PY{l+s+s2}{\PYZdq{}}\PY{l+s+s2}{ }\PY{l+s+s2}{\PYZdq{}}\PY{p}{)}\PY{p}{[}\PY{l+m+mi}{0}\PY{p}{]}\PY{p}{)}
                 \PY{n}{board}\PY{o}{.}\PY{n}{push}\PY{p}{(}\PY{n}{get\PYZus{}random\PYZus{}move}\PY{p}{(}\PY{n}{board}\PY{p}{)}\PY{p}{)}
             \PY{k}{return} \PY{n}{turn\PYZus{}list}
\end{Verbatim}


    Nach Ende des Spiels wird über der zu spielenden Farbe ermittelt, wer
der Sieger des Spiels ist und dementsprechend der Wert "+1" oder "-1"
zurückgegeben. Mit diesem wird dann eine Key-Value Liste erstellt, die
zu jedem Spielbrett des vergangenen Spiels den Wert zuweist, der Aussage
über den Sieger gibt.

Diese Liste wird dann zusammen gefügt mit der bisher vorhandenen
Historie. Falls bereits ein Eintrag für ein gleichartiges Schachbrett
existiert, wird der Siegwert zu dem aktuellen Wert aus der Historie dazu
addiert, andernfalls wird ein neuer Eintrag mit dem Siegwert angelegt.

    \begin{Verbatim}[commandchars=\\\{\}]
{\color{incolor}In [{\color{incolor}17}]:} \PY{k}{def} \PY{n+nf}{store\PYZus{}game}\PY{p}{(}\PY{n}{turn\PYZus{}list}\PY{p}{,} \PY{n}{victory\PYZus{}status}\PY{p}{)}\PY{p}{:}
                 \PY{n}{new\PYZus{}turn\PYZus{}dict} \PY{o}{=} \PY{n+nb}{dict}\PY{o}{.}\PY{n}{fromkeys}\PY{p}{(}\PY{n}{turn\PYZus{}list}\PY{p}{,} \PY{n}{victory\PYZus{}status}\PY{p}{)}
                 \PY{c+c1}{\PYZsh{} get existing board history}
                 \PY{n}{history} \PY{o}{=} \PY{n}{pd}\PY{o}{.}\PY{n}{read\PYZus{}csv}\PY{p}{(}\PY{n}{HISTORY\PYZus{}FILE\PYZus{}LOC}\PY{p}{)}
                 \PY{n}{history\PYZus{}dict} \PY{o}{=} \PY{n+nb}{dict}\PY{p}{(}\PY{n+nb}{zip}\PY{p}{(}\PY{n+nb}{list}\PY{p}{(}\PY{n}{history}\PY{o}{.}\PY{n}{board}\PY{p}{)}\PY{p}{,} \PY{n+nb}{list}\PY{p}{(}\PY{n}{history}\PY{o}{.}\PY{n}{value}\PY{p}{)}\PY{p}{)}\PY{p}{)}
                 \PY{c+c1}{\PYZsh{} merge existing history with new boards and sum the victory states}
                 \PY{n}{merged\PYZus{}history\PYZus{}dict} \PY{o}{=} \PY{p}{\PYZob{}} \PY{n}{k}\PY{p}{:} \PY{n}{new\PYZus{}turn\PYZus{}dict}\PY{o}{.}\PY{n}{get}\PY{p}{(}\PY{n}{k}\PY{p}{,} \PY{l+m+mi}{0}\PY{p}{)} \PY{o}{+} \PY{n}{history\PYZus{}dict}\PY{o}{.}\PY{n}{get}\PY{p}{(}\PY{n}{k}\PY{p}{,} \PY{l+m+mi}{0}\PY{p}{)} \PY{k}{for} \PY{n}{k} \PY{o+ow}{in} \PY{n+nb}{set}\PY{p}{(}\PY{n}{new\PYZus{}turn\PYZus{}dict}\PY{p}{)} \PY{o}{|} \PY{n+nb}{set}\PY{p}{(}\PY{n}{history\PYZus{}dict}\PY{p}{)} \PY{p}{\PYZcb{}}
                 \PY{n}{merged\PYZus{}history} \PY{o}{=} \PY{n}{pd}\PY{o}{.}\PY{n}{DataFrame}\PY{p}{(}\PY{n+nb}{list}\PY{p}{(}\PY{n}{merged\PYZus{}history\PYZus{}dict}\PY{o}{.}\PY{n}{items}\PY{p}{(}\PY{p}{)}\PY{p}{)}\PY{p}{,} \PY{n}{columns}\PY{o}{=}\PY{p}{[}\PY{l+s+s1}{\PYZsq{}}\PY{l+s+s1}{board}\PY{l+s+s1}{\PYZsq{}}\PY{p}{,}\PY{l+s+s1}{\PYZsq{}}\PY{l+s+s1}{value}\PY{l+s+s1}{\PYZsq{}}\PY{p}{]}\PY{p}{)}
                 \PY{c+c1}{\PYZsh{} overwrite history csv with new, merged history}
                 \PY{n}{merged\PYZus{}history}\PY{o}{.}\PY{n}{to\PYZus{}csv}\PY{p}{(}\PY{n}{HISTORY\PYZus{}FILE\PYZus{}LOC}\PY{p}{)}
\end{Verbatim}


    Der Siegwert nach dem Schachbrett gibt dann, je nach Höhe, Auskunft
darüber, wie wahrscheinlich es ist mit einem solchen Schachbrett zu
gewinnen. Umso höher der Wert in den positiven Bereich fällt, desto
wahrscheinlicher ist ein Sieg für Spieler 1. Umso höher der Wert in den
negative Bereich fällt, desto besser sind die Aussichten für Spieler 2.

Diese Erkenntnis kann dann von der KI genutzt werde, um beim "iterative
Deepening" Prozess alle möglichen Schachbretter zu evaluieren und sich
so für den besten Zug zu entscheiden.

Der folgende Code koordiniert die Abläufe der Funktionen um die Datei
\texttt{history.csv} zu erweitern.

    \begin{Verbatim}[commandchars=\\\{\}]
{\color{incolor}In [{\color{incolor} }]:} \PY{k}{for} \PY{n}{i} \PY{o+ow}{in} \PY{n+nb}{range}\PY{p}{(}\PY{l+m+mi}{0}\PY{p}{,} \PY{l+m+mi}{10}\PY{p}{)}\PY{p}{:}
            \PY{n}{i} \PY{o}{+}\PY{o}{=} \PY{l+m+mi}{1}
            \PY{n}{board} \PY{o}{=} \PY{n}{chess}\PY{o}{.}\PY{n}{Board}\PY{p}{(}\PY{p}{)}
            \PY{n}{turn\PYZus{}list} \PY{o}{=} \PY{n}{play\PYZus{}random\PYZus{}chess\PYZus{}game}\PY{p}{(}\PY{n}{board}\PY{p}{)}
        
            \PY{n}{victory\PYZus{}status} \PY{o}{=} \PY{l+m+mi}{1} \PY{k}{if} \PY{n}{board}\PY{o}{.}\PY{n}{turn} \PY{k}{else} \PY{o}{\PYZhy{}}\PY{l+m+mi}{1}
            \PY{n}{store\PYZus{}game}\PY{p}{(}\PY{n}{turn\PYZus{}list}\PY{p}{,} \PY{n}{victory\PYZus{}status}\PY{p}{)}
        \PY{n+nb}{print}\PY{p}{(}\PY{l+s+s2}{\PYZdq{}}\PY{l+s+s2}{Games finished}\PY{l+s+s2}{\PYZdq{}}\PY{p}{)}
\end{Verbatim}


    Ein Ausschnitt aus der history.csv ist im folgenden Snippet zu sehen.

    \begin{Verbatim}[commandchars=\\\{\}]
{\color{incolor}In [{\color{incolor} }]:} \PY{n}{df} \PY{o}{=} \PY{n}{pd}\PY{o}{.}\PY{n}{read\PYZus{}csv}\PY{p}{(}\PY{n}{HISTORY\PYZus{}FILE\PYZus{}LOC}\PY{p}{)}
        \PY{n}{df}\PY{o}{.}\PY{n}{head}\PY{p}{(}\PY{p}{)}
\end{Verbatim}


    Damit eingeschätzt werden kann, wie aussichtsreich ein bestimmter
Boardzustand ist, kann ein gespeicherter Spielverlauf zur Hilfe genommen
werden. Mit Hilfe der Funktion \texttt{compare\_board\_history} kann ein
Erwartungswert berechnet werden. Umso höher der zurückgegebene Wert der
Funktion, desto wahrscheinlicher ist ein Sieg für Weiß. Hingegen ist ein
Sieg für Schwarz umso wahrscheinlicher, sobald der Wert deutlicher in
den negative Bereich fällt.

Der Vorgang zur Berechnung dieses Wertes beginnt mit dem Einlesen der
\texttt{history.csv} in ein Pandas Dataframe. Daraufhin wird durch die
Reihe des Dataframes iteriert, die die Schachbretter in FEN-Schreibweise
enthält. Sobald die aktuelle Spielsituation in der Historie gefunden
wurde, wird deren Wert ausgelesen und zurückgegeben. Falls die aktuelle
Spielsituation nicht vorhanden ist, wird der Wert \texttt{0}
zurückgegeben.

    \begin{Verbatim}[commandchars=\\\{\}]
{\color{incolor}In [{\color{incolor} }]:} \PY{k}{def} \PY{n+nf}{compare\PYZus{}board\PYZus{}history}\PY{p}{(}\PY{n}{board}\PY{p}{)}\PY{p}{:}
                \PY{n}{df} \PY{o}{=} \PY{n}{pd}\PY{o}{.}\PY{n}{read\PYZus{}csv}\PY{p}{(}\PY{n}{HISTORY\PYZus{}FILE\PYZus{}LOC}\PY{p}{)}
                \PY{n}{row} \PY{o}{=} \PY{n}{df}\PY{o}{.}\PY{n}{loc}\PY{p}{[}\PY{n}{df}\PY{p}{[}\PY{l+s+s1}{\PYZsq{}}\PY{l+s+s1}{board}\PY{l+s+s1}{\PYZsq{}}\PY{p}{]} \PY{o}{==} \PY{n}{board}\PY{o}{.}\PY{n}{fen}\PY{p}{(}\PY{p}{)}\PY{o}{.}\PY{n}{split}\PY{p}{(}\PY{l+s+s2}{\PYZdq{}}\PY{l+s+s2}{ }\PY{l+s+s2}{\PYZdq{}}\PY{p}{)}\PY{p}{[}\PY{l+m+mi}{0}\PY{p}{]}\PY{p}{]}
                \PY{n}{value} \PY{o}{=} \PY{n}{row}\PY{p}{[}\PY{l+s+s1}{\PYZsq{}}\PY{l+s+s1}{value}\PY{l+s+s1}{\PYZsq{}}\PY{p}{]}\PY{o}{.}\PY{n}{item}\PY{p}{(}\PY{p}{)} \PY{k}{if} \PY{n+nb}{len}\PY{p}{(}\PY{n}{row}\PY{p}{[}\PY{l+s+s1}{\PYZsq{}}\PY{l+s+s1}{value}\PY{l+s+s1}{\PYZsq{}}\PY{p}{]}\PY{p}{)} \PY{o}{==} \PY{l+m+mi}{1} \PY{k}{else} \PY{l+m+mi}{0}
                \PY{k}{return} \PY{n}{value}
\end{Verbatim}


    \subsection{Überprüfung auf ein Schach oder
Spielende}\label{uxfcberpruxfcfung-auf-ein-schach-oder-spielende}

Damit der Spielverlauf korrekt nach den Regeln des Spiels verlaufen
kann, muss während jedes Zuges darauf geprüft werden, ob ein Schach,
Schachmatt oder Patt vorliegt. Schach ist hierbei eine Stellung während
des Spiels, bei dem der König in Bedrängnis geraten ist. Ein Schach kann
im weiteren Verlauf zu einem Schachmatt führen. Hierbei liegt der
Unterschied darin, dass der Spieler, der sich im Schachmatt befindet mit
keinem regelkonformen Zug sich aus dem Schachmatt befreien kann und
somit das Spiel verloren hat. Steht der König Schach kann er mit einem
gezielten Zug sich aus dieser Lage befreien. Ein Patt ist eine
Endposition beim Schach, die die Eigenschaft hat, dass keiner der beiden
Spieler ein Schachmatt erreichen kann.

Da ein Schachmatt oder Patt das Schachspiel beendet ist diese Prüfung
ein essenzieller Aspekt der zu entwickelnden KI. Um diese Prüfung
durchzuführen bietet die verwendete Bibliothek bereits einige
Funktionen. Ebenso ist es wichtig zu überprüfen, ob ein Schach vorliegt,
da der Spieler zuerst dies lösen muss, bevor er weiterspielen kann. Um
diese Prüfung durchzuführen steht jeweils für die Überprüfung eines
Schachs oder Spielendes eine Funktion zur Verfügung, die das aktuelle
Spiel auf diesen Aspekt überprüft. Um herauszufinden, ob es sich um ein
Schach handelt, kann die folgende Funktion verwendet werden. *
\texttt{is\_check()} * Überprüft die aktuellen Begebenheiten auf ein
mögliches Schach

Hingegen gibt es die Funktion \texttt{is\_game\_over}, die die
gespielten Züge auf jegliche Arten einer benötigten Beendigung des
Spiels überprüft. Hierbei beinhaltet \texttt{is\_game\_over} einige
mögliche Überprüfungen auf Schachmatt oder Patt. *
\texttt{is\_game\_over()} * Stellt sicher, ob das Spiel auf Grund eines
Schachmatts oder anderer spielbeendender Umstände vorbei ist * Hierbei
wird überprüft, ob ein Schachmatt (\texttt{is\_checkmate()}), ein Patt
(\texttt{is\_stalemate()}), eine Tote Stellung
(\texttt{is\_insufficient\_material()}), die 75-Züge-Regel
(\texttt{is\_seventyfive\_moves()}), eine Figur fünfmal auf der gleichen
Position sich befindet (\texttt{fivefold\_repetition}), oder eine
spezielle Endbedingung vorliegt.

Um herauszufinden, ob ein Spieler tatsächlich gewonnen hat kann eine
Teilfunktion der zuvor genannten genutzt werden. Diese Funktion
überprüft lediglich, ob ein Schachmatt vorliegt. *
\texttt{is\_checkmate()} * Stellt fest, ob es sich nach den bereits
gespielten Zügen ein Schachmatt vorliegt

    \begin{Verbatim}[commandchars=\\\{\}]
{\color{incolor}In [{\color{incolor} }]:} \PY{c+c1}{\PYZsh{} import situation where checkmate is True}
        \PY{n}{board} \PY{o}{=} \PY{n}{chess}\PY{o}{.}\PY{n}{Board}\PY{p}{(}\PY{l+s+s2}{\PYZdq{}}\PY{l+s+s2}{r1bqkb1r/pppp1Qpp/2n2n2/4p3/2B1P3/8/PPPP1PPP/RNB1K1NR b KQkq \PYZhy{} 0 4}\PY{l+s+s2}{\PYZdq{}}\PY{p}{)}
        
        \PY{n+nb}{print}\PY{p}{(}\PY{l+s+s2}{\PYZdq{}}\PY{l+s+s2}{Check: }\PY{l+s+s2}{\PYZdq{}}\PY{p}{,} \PY{n}{board}\PY{o}{.}\PY{n}{is\PYZus{}check}\PY{p}{(}\PY{p}{)}\PY{p}{)}
        \PY{n+nb}{print}\PY{p}{(}\PY{l+s+s2}{\PYZdq{}}\PY{l+s+s2}{Checkmate: }\PY{l+s+s2}{\PYZdq{}}\PY{p}{,} \PY{n}{board}\PY{o}{.}\PY{n}{is\PYZus{}checkmate}\PY{p}{(}\PY{p}{)}\PY{p}{)}
        \PY{n+nb}{print}\PY{p}{(}\PY{l+s+s2}{\PYZdq{}}\PY{l+s+s2}{Game is over: }\PY{l+s+s2}{\PYZdq{}}\PY{p}{,} \PY{n}{board}\PY{o}{.}\PY{n}{is\PYZus{}game\PYZus{}over}\PY{p}{(}\PY{p}{)}\PY{p}{)}
\end{Verbatim}


    Im folgenden wird ein Beispiel einer Spielsituation gezeigt, in der ein
Schachmatt vorliegt und die auf Schach, Schachmatt und Spielende
überprüft wurde.

    \begin{Verbatim}[commandchars=\\\{\}]
{\color{incolor}In [{\color{incolor} }]:} \PY{n}{SVG}\PY{p}{(}\PY{n}{chess}\PY{o}{.}\PY{n}{svg}\PY{o}{.}\PY{n}{board}\PY{p}{(}\PY{n}{board}\PY{o}{=}\PY{n}{board}\PY{p}{)}\PY{p}{)}
\end{Verbatim}


    \subsection{Überprüfung, ob "en passant" und "Rochaden" unterstützt
werden}\label{uxfcberpruxfcfung-ob-en-passant-und-rochaden-unterstuxfctzt-werden}

Im Schach gibt es einige spezielle Züge, die es ermöglichen eine Figur
eine Aktion durchzuführen zu lassen, die normalerweise laut der
grundlegenden Definition dieser nicht möglich ist.

Die Schachfigur eines Bauers darf normalerweise nur dann eine andere
Figur schlagen, wenn sich diese direkt in dem diagonal vor dem Bauern
angrenzenden Feld befindet. Durch die sogenannte Regel "en passant", im
deutschen "im Vorbeigehen", wird diese Regel erweitert. Eingesetzt
werden kann "en passant", wenn auf einen Bauern die Sonderregel des
Doppelschritts aus der Grundstellung angewendet wird. Steht in diesem
Fall der mit Doppelschritt herausgerückte Bauern neben einem des
Gegners, dann kann dieser den neu herausgerückten Bauern durch "en
passant" schlagen. Hierbei springt der angreifende Bauer des Gegners
direkt hinter den herausgerückten und schlägt ihn somit.

Damit die Schach-KI alle möglichen Züge des Gegners bedenken und ebenso
alle Züge ausführen soll, muss überprüft werden, ob die verwendete
Library die beiden Sonderzüge unterstützt, oder ob diese Unterstützung
manuell entwickelt werden muss.

    \begin{Verbatim}[commandchars=\\\{\}]
{\color{incolor}In [{\color{incolor} }]:} \PY{c+c1}{\PYZsh{} import situation where en passant is possible}
        \PY{n}{board} \PY{o}{=} \PY{n}{chess}\PY{o}{.}\PY{n}{Board}\PY{p}{(}\PY{l+s+s2}{\PYZdq{}}\PY{l+s+s2}{rnbqkbnr/1pp1pppp/p7/3pP3/8/8/PPPP1PPP/RNBQKBNR w KQkq d6 0 3}\PY{l+s+s2}{\PYZdq{}}\PY{p}{)}
        
        \PY{c+c1}{\PYZsh{} the function board.has\PYZus{}legal\PYZus{}en\PYZus{}passant() could check if a en passant is possible, but does not return at which square}
        \PY{n+nb}{print}\PY{p}{(}\PY{l+s+s2}{\PYZdq{}}\PY{l+s+s2}{Library is supporting en passant: }\PY{l+s+s2}{\PYZdq{}}\PY{p}{,}\PY{n}{chess}\PY{o}{.}\PY{n}{Move}\PY{o}{.}\PY{n}{from\PYZus{}uci}\PY{p}{(}\PY{l+s+s2}{\PYZdq{}}\PY{l+s+s2}{e5d6}\PY{l+s+s2}{\PYZdq{}}\PY{p}{)} \PY{o+ow}{in} \PY{n}{board}\PY{o}{.}\PY{n}{legal\PYZus{}moves}\PY{p}{)}
\end{Verbatim}


    Die Chess-Core Bibliothek unterstützt somit den Sonderzug en passent.

Im Folgenden wird ein Schachbrett angezeigt, bei dem es dem weißen Bauer
(\texttt{e5}) möglich ist im Vorübergehen den schwarzen Bauer
(\texttt{d5}) zu schlagen.

    \begin{Verbatim}[commandchars=\\\{\}]
{\color{incolor}In [{\color{incolor} }]:} \PY{n}{squares} \PY{o}{=} \PY{n}{chess}\PY{o}{.}\PY{n}{SquareSet}\PY{p}{(}\PY{p}{[}\PY{n}{chess}\PY{o}{.}\PY{n}{D6}\PY{p}{]}\PY{p}{)}
        \PY{n}{chess}\PY{o}{.}\PY{n}{svg}\PY{o}{.}\PY{n}{board}\PY{p}{(}\PY{n}{board}\PY{o}{=}\PY{n}{board}\PY{p}{,} \PY{n}{squares}\PY{o}{=}\PY{n}{squares}\PY{p}{)} 
\end{Verbatim}


    Neben dem en passant gibt es einen weiteren bekannten Sonderzug, die
sogenannte Rochade. Bei der Rochade lassen sich die Positionen eines
Turms und des Königs tauschen, wobei für diesen Vorgang nur ein Zug
benötigt wird. Dabei ist zu beachten, dass die Voraussetzung für diesen
Zug ist, dass sowohl der zu verwendende Turm, als auch der König im
Verlauf des Spiels nicht genutzt wurden. Ebenfalls dürfen die Felder
zwischen König und Turm nicht belegt sein und keines der Felder, durch
die der König ziehen muss, darf durch eine gegnerische Figur bedroht
sein, sowie der König vor und nach der Rochade nicht im Schach stehen.

Für jeden Spieler gibt es zwei verschiedene Möglichkeiten der Rochade,
einerseits die kurze, als auch die lange Rochade. Ein Beispiel für eine
lange Rochade der weißen Figuren ist, dass der Turm (\texttt{a1}) und
der König (\texttt{e1}) ihre Positionen tauschen und somit der Turm sich
auf dem Feld \texttt{d1} und der König auf \texttt{c1} befindet.

    \begin{Verbatim}[commandchars=\\\{\}]
{\color{incolor}In [{\color{incolor} }]:} \PY{c+c1}{\PYZsh{} import situation where castling is possible}
        \PY{n}{board} \PY{o}{=} \PY{n}{chess}\PY{o}{.}\PY{n}{Board}\PY{p}{(}\PY{l+s+s2}{\PYZdq{}}\PY{l+s+s2}{8/8/8/8/8/8/8/R3K3 w KQkq \PYZhy{} 0 1}\PY{l+s+s2}{\PYZdq{}}\PY{p}{)}
        \PY{c+c1}{\PYZsh{} shortcut for the castling\PYZhy{}move with the queenside rook}
        \PY{n}{castling} \PY{o}{=} \PY{l+s+s2}{\PYZdq{}}\PY{l+s+s2}{O\PYZhy{}O\PYZhy{}O}\PY{l+s+s2}{\PYZdq{}}
        
        \PY{c+c1}{\PYZsh{} check if castling is in the legal moves and print the result}
        \PY{k}{if} \PY{n}{castling} \PY{o+ow}{in} \PY{n+nb}{str}\PY{p}{(}\PY{n}{board}\PY{o}{.}\PY{n}{legal\PYZus{}moves}\PY{p}{)}\PY{p}{:}
            \PY{n+nb}{print}\PY{p}{(}\PY{l+s+s2}{\PYZdq{}}\PY{l+s+s2}{Library is supporting castling: True}\PY{l+s+s2}{\PYZdq{}}\PY{p}{)}
            \PY{n}{board}\PY{o}{.}\PY{n}{push\PYZus{}san}\PY{p}{(}\PY{n}{castling}\PY{p}{)}
        \PY{k}{else}\PY{p}{:}
            \PY{n+nb}{print}\PY{p}{(}\PY{l+s+s2}{\PYZdq{}}\PY{l+s+s2}{Library is supporting castling: False}\PY{l+s+s2}{\PYZdq{}}\PY{p}{)}
        
        \PY{n}{SVG}\PY{p}{(}\PY{n}{chess}\PY{o}{.}\PY{n}{svg}\PY{o}{.}\PY{n}{board}\PY{p}{(}\PY{n}{board}\PY{o}{=}\PY{n}{board}\PY{p}{)}\PY{p}{)} 
\end{Verbatim}


    Eine durchgeführte kurze weiße Rochade, bei der \texttt{a1} und
\texttt{e1} die Positionen getauscht haben, sodass der König sich nun
auf \texttt{c1} und der Turm auf \texttt{d1} befindet. Die Abkürzung für
eine kurze Rochade ist \emph{O-O}, das für den Turm auf der Seite des
Königs steht und \emph{O-O-O} für eine lange Rochade, wobei die
Abkürzung für den Turm auf der Seite der Dame steht.

    \subsection{Einbinden eines Opening-Books in die Chess
AI}\label{einbinden-eines-opening-books-in-die-chess-ai}

Polyglot ist ein Open-Source Format, in dem sogenannte Opening-Books
erstellt werden können. Opening-Books sind Ansammlungen von Spielzügen,
die im Schach zur Eröffnung genutzt werden können. Hierbei wird zum
Einstieg des Spiels nicht auf eine Künstliche Intelligenz
zurückgegriffen, sondern auf ein Verzeichnis von bereits bestehenden
Zügen, die sich innerhalb dieses Opening-Books befinden. Der Vorteil
eines Opening-Books liegt darin, dass bereits qualitativ hochwertige
Strategien verfügbar sind und somit ein anspruchsvolles Spiel
ermöglichen.

Sobald das Opening-Book, im Verlauf des Spiels, keinen passenden Zug als
Antwort bereitstellen kann, übernimmt die Künstliche Intelligenz.

Für das Einbinden eines Opening-Books werden einige Funktionen aus der
Core Chess Library genutzt und müssen somit eingebunden werden.

    \begin{Verbatim}[commandchars=\\\{\}]
{\color{incolor}In [{\color{incolor} }]:} \PY{k+kn}{import} \PY{n+nn}{chess}\PY{n+nn}{.}\PY{n+nn}{polyglot}
\end{Verbatim}


    Als nächstes wird ein neues Spiel/Board erzeugt, indem die Figuren
standardmäßig angeordnet werden. Ebenfalls wird ein Opening-Book in eine
Variable geladen, damit aus dieser der bestmöglichen Zug ausgewählt
werden kann. Dies geschieht, indem das Opening-Book das aktuelle Board
übergeben bekommt und anhand der Positionen der Figuren einen passenden
Zug auswählt.

Der ausgewählte Zug wird als nächstes auf dem Schachbrett ausgeführt und
somit wird in diesem Beispiel der weiße Bauer von \texttt{e2} nach
\texttt{e4} gezogen.

    \begin{Verbatim}[commandchars=\\\{\}]
{\color{incolor}In [{\color{incolor} }]:} \PY{n}{board} \PY{o}{=} \PY{n}{chess}\PY{o}{.}\PY{n}{Board}\PY{p}{(}\PY{p}{)}
        \PY{n}{book} \PY{o}{=} \PY{n}{chess}\PY{o}{.}\PY{n}{polyglot}\PY{o}{.}\PY{n}{open\PYZus{}reader}\PY{p}{(}\PY{l+s+s2}{\PYZdq{}}\PY{l+s+s2}{res/polyglot/Performance.bin}\PY{l+s+s2}{\PYZdq{}}\PY{p}{)}
        
        \PY{k}{def} \PY{n+nf}{get\PYZus{}move}\PY{p}{(}\PY{n}{board}\PY{p}{)}\PY{p}{:}
            \PY{c+c1}{\PYZsh{} find the move with the highest weight for the current board}
            \PY{k}{try}\PY{p}{:}
                \PY{n}{main\PYZus{}entry} \PY{o}{=} \PY{n}{book}\PY{o}{.}\PY{n}{find}\PY{p}{(}\PY{n}{board}\PY{p}{)}
                \PY{n}{all\PYZus{}entries} \PY{o}{=} \PY{n}{book}\PY{o}{.}\PY{n}{find\PYZus{}all}\PY{p}{(}\PY{n}{board}\PY{p}{)}
        
                \PY{n}{move} \PY{o}{=} \PY{n}{main\PYZus{}entry}\PY{o}{.}\PY{n}{move}\PY{p}{(}\PY{p}{)}
                \PY{n+nb}{print}\PY{p}{(}\PY{l+s+s2}{\PYZdq{}}\PY{l+s+s2}{Selected move with the highest weight: }\PY{l+s+s2}{\PYZdq{}}\PY{p}{,} \PY{n}{move}\PY{p}{)}
                \PY{n+nb}{print}\PY{p}{(}\PY{l+s+s2}{\PYZdq{}}\PY{l+s+s2}{All available moves for this situation: }\PY{l+s+s2}{\PYZdq{}}\PY{p}{,} \PY{l+s+s2}{\PYZdq{}}\PY{l+s+s2}{, }\PY{l+s+s2}{\PYZdq{}}\PY{o}{.}\PY{n}{join}\PY{p}{(}\PY{p}{[}\PY{n+nb}{str}\PY{p}{(}\PY{n}{entry}\PY{o}{.}\PY{n}{move}\PY{p}{(}\PY{p}{)}\PY{p}{)} \PY{k}{for} \PY{n}{entry} \PY{o+ow}{in} \PY{n}{all\PYZus{}entries}\PY{p}{]}\PY{p}{)}\PY{p}{)}
        
                \PY{k}{return} \PY{n}{move}
            \PY{k}{except} \PY{n+ne}{IndexError}\PY{p}{:}
                \PY{n+nb}{print}\PY{p}{(}\PY{l+s+s2}{\PYZdq{}}\PY{l+s+s2}{The opening book cannot find an appropriate move!}\PY{l+s+s2}{\PYZdq{}}\PY{p}{)}
        
        \PY{k}{for} \PY{n}{i} \PY{o+ow}{in} \PY{n+nb}{range} \PY{p}{(}\PY{l+m+mi}{1}\PY{p}{,}\PY{l+m+mi}{3}\PY{p}{)}\PY{p}{:}
            \PY{n}{next\PYZus{}move} \PY{o}{=} \PY{n}{get\PYZus{}move}\PY{p}{(}\PY{n}{board}\PY{p}{)}
            \PY{n}{board}\PY{o}{.}\PY{n}{push}\PY{p}{(}\PY{n}{next\PYZus{}move}\PY{p}{)}
            \PY{n+nb}{print}\PY{p}{(}\PY{l+s+s2}{\PYZdq{}}\PY{l+s+s2}{Selected move: }\PY{l+s+si}{\PYZob{}\PYZcb{}}\PY{l+s+se}{\PYZbs{}n}\PY{l+s+s2}{\PYZdq{}}\PY{o}{.}\PY{n}{format}\PY{p}{(}\PY{n}{next\PYZus{}move}\PY{p}{)}\PY{p}{)}
        \PY{n}{book}\PY{o}{.}\PY{n}{close}\PY{p}{(}\PY{p}{)}
        
        \PY{n}{SVG}\PY{p}{(}\PY{n}{chess}\PY{o}{.}\PY{n}{svg}\PY{o}{.}\PY{n}{board}\PY{p}{(}\PY{n}{board}\PY{o}{=}\PY{n}{board}\PY{p}{)}\PY{p}{)}
\end{Verbatim}



    % Add a bibliography block to the postdoc
    
    
    
    \end{document}
